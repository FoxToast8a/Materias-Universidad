\input{encabezado}
\usepackage{amsmath}
\usepackage{geometry}
\usepackage{tikz}
\usepackage{float}
\usepackage{graphics}

\tikzset{every picture/.style={line width=0.75pt}} %set default line width to 0.75pt        

\begin{document}
\maketitle
\thispagestyle{empty}
\newpage 

\begin{homeworkProblem}
    Pruebe que los siguientes conjuntos no son algebraicos:
    \begin{itemize}
        \item $\{(z,w)\in A^2(\mathbb{C})\,|\,|z|^2+|w|^2=1\}.$ 
            \begin{solucion}
                Sea $X=\{(z,w)\in A^2(\mathbb{C})\,|\,|z|^2+|w|^2=1\}.$ Suponga que $X$ es un conjunto algebraico, luego existe $ S\subseteq \mathbb{C}[x,y]$, no vació y de polinomios no constantes tal que $X=V(S).$ Sea $f(x,y)=\sum_{i=0}^{m}f_i(y)x^i\in S,$ donde $f_i(y)\in\mathbb{C}[y].$ Consideremos $w_0\in\mathbb{C}$ tal que $|w_0|<1,$ luego $|z|^2=1-|w_0|^2$, luego el polinomio $f(x,w_0)=\sum_{i=0}^{m}f_i(w_0)x^i$ tiene infinitas raíces, ya que hay infinitos puntos en el circulo del plano complejo centrado en $0$ y de radio $1-|w_0|^2.$ Pero esto solo es posible si $f_i(w_0)=0$ para todo $|w_0|<1$ y cada $i=0,1,\ldots,m$, luego como hay infinitos $w_0$ complejos, que cumplen tener norma menor que 1, $f_i=0$ para todo $i=0,1,\ldots,m$, así $f=0,$ pero esto es una contradicción, ya que $f$ por hipótesis es no constante. Así concluimos que $X$ no es algebraico.

                \qed
            \end{solucion}
        \item $\{(\cos(t),\sin(t),t)\in A^3(\mathbb{R})\,|\,t\in\mathbb{R}\}.$
            \begin{solucion}
                Sea $X=\{(\cos(t),\sin(t),t)\in A^3(\mathbb{R})\,|\,t\in\mathbb{R}\}.$ Suponga que $X$ es un conjunto algebraico, luego existe $S\subseteq\mathbb{R}[x,y,z]$, no vació y de polinomios no constantes tal que $X=V(S).$ Sea $f(x,y,z)=\sum_{i=0}^{m}f_i(x,y)z^i\in S,$ donde $f_i(x,y)\in \mathbb{R}[x,y].$ Consideremos $\theta_0\in[0,2\pi)$ luego por la periodicidad del coseno y del seno, note que $f(\cos(\theta_0+2k\pi),\sin(\theta_0+2k\pi),\theta_0+2k\pi)=f(\cos(\theta_0),\sin(\theta_0),\theta_0+2k\pi)=0,$ para todo $k\in\mathbb{Z},$ ya que este ultimo es un punto de $X.$ Así para $\theta_0$ fijo tenemos que $f(\cos(\theta_0),\sin(\theta_0),z)$ tiene infinitas raíces, luego $f_i(\cos(\theta_0),\sin(\theta_0))=0$ para todo $\theta_0\in[0,2\pi)$ y cada $i=0,1,\ldots,m.$ Así cada $f_i$ tiene infinitas raíces, entonces $f_i=0$ para todo $i=0,1,\ldots,m$, luego $f=0,$ llegando así a una contradicción y concluyendo que $X$ no es algebraico.

                \qed 
            \end{solucion}
    \end{itemize}
\end{homeworkProblem}
\newpage
\begin{homeworkProblem}
    Muestre a través de un ejemplo que la unión infinita de algebraicos no siempre es un conjunto algebraico
    \begin{solucion}
        Consideremos $A^1(\R)=\R$ y tomemos $X= \Z \subseteq A^1(\R)$ note que $\Z=\bigcup_{a\in Z}\{a\},$ sabemos que $\{a\}=V(x-a)$, por lo que $\Z$ es la unión contable de conjuntos algebraicos. Si $X$ fuera algebraico existiría $S\subseteq \R[x]$, no vació y de polinomios no constantes, tal que $X=V(S)$, luego dado $f\in S,$ es un polinomio con infinitas raíces ya que todos los enteros son raíces, así $f=0,$ una contradicción. Mostrando así que una unión infinita de conjuntos algebraicos no es un conjunto algebraico siempre.

        \qed
    \end{solucion}
\end{homeworkProblem}

\begin{homeworkProblem}
    Sean $V\subseteq A^n(\K)$ y $W\subseteq A^m(\K)$ conjuntos algebraicos. Demuestre que
    $$V\times W=\{(a_1,\ldots,a_n,b_1,\ldots,b_m)\in A^{n+m}(\K)\,|\,(a_1,\ldots,a_n)\in V,(b_1,\ldots,b_m)\in W\}$$
    es algebraico.
    \begin{solucion}
        Como $V$ y $W$ son algebraicos, existen $S_1\subseteq \K[x_1,\ldots,x_n]$ y $S_2\subseteq \K[y_1,\ldots,y_m],$ no vacíos y de polinomios no constantes tales que $V=V(S_1)$ y $W=V(S_2).$ Definamos los conjuntos $\widehat{S}_1,\widehat{S}_2\subseteq \K[x_1,\ldots,x_n,y_1,\ldots,y_n],$ tales que, dada $\hat{f}\in \widehat{S}_1$, tenemos que $\hat{f}(x_1,\ldots,x_n,y_1,\ldots,y_n)=f(x_1,\ldots,x_n),$ donde $f\in S_1,$ y dada $\hat{g}\in\widehat{S}_2$, tenemos que $\hat{g}(x_1,\ldots,x_n,y_1,\ldots,y_n)=g(y_1,\ldots,y_n),$ donde $g\in S_2.$ Basta con probar que $V\times W=V(\widehat{S}_1\cup\widehat{S}_2).$ Sea $p\in V\times W,$ luego $p=(a_1,\ldots,a_n,b_1,\ldots,b_m),$ donde $(a_1,\ldots,a_n)\in V$ y $(b_1,\ldots,b_m)\in W,$ Note que tomando $h\in \widehat{S}_1\cup\widehat{S}_2 $ tenemos que $h=\hat{f}$ o $h=\hat{g}$, para algún $\hat{f}\in \widehat{S}_1$ o $\hat{g}\in\widehat{S}_2.$\\
        En el primer caso tenemos que $h(p)=\hat{f}(p)=f(a_1,\ldots,a_n)=0,$ ya que $(a_1,\ldots,a_n)\in V(S_1).$ En el otro caso tenemos que $h(p)=\hat{g}(p)=g(b_1,\ldots,b_m)=0,$ ya que $(b_1,\ldots,b_m)\in V(S_2).$ Concluyendo así que para todo $h\in widehat{S}_1\cup\widehat{S}_2$ tenemos que $h(p)=0$ y por tanto $p\in V(\widehat{S}_1\cup\widehat{S}_2),$ mostrando así que $V\times W\subseteq V(\widehat{S}_1\cup\widehat{S}_2).$ Ahora sea $p\in V(\widehat{S}_1\cup\widehat{S}_2)$ luego para todo $h\in \widehat{S}_1\cup\widehat{S}_2 .$ $h(p)=0.$ Para los $h\in \widehat{S}_1 $ tenemos que $h(p)=\hat{f}(p)=f(p_1,\ldots,p_n).$ Como es arbitrario, tenemos que para todo $f\in S_1,$ $f(p_1,\ldots,p_n)=0,$ es decir, $(p_1,\ldots,p_n)\in V(S_1)=V.$ De manera similar para los $h\in \widehat{S}_2,$ tenemos que $0=h(p)=\hat{g}(p)=g(p_{n+1},\ldots,p_{n+m}),$ concluyendo análogamente que $(p_{n+1},\ldots,p_{n+m})\in V(S_2)=W.$ Luego por como lo definimos concluimos que $p\in V\times W,$ mostrando que $V(\widehat{S}_1\cup\widehat{S}_2) \subseteq V\times W.$ Por la doble continencia, concluimos la igualdad y por tanto que $V\times W$ es algebraico. \qed
    \end{solucion}
\end{homeworkProblem}

%%%%%%%%%%%%%%%%%%%%%%%%%%%%%%%%%%%%%%%%%%%%%%%%%%%%%%%
\end{document}